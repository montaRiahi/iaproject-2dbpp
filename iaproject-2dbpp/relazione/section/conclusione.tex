Dall'esperienza accumulata sviluppando questo progetto è stato valutato che gli algoritmi metaeuristici, nella fattispecie \emph{genetico} e \emph{tabu search}, si comportano bene nella soluzione del \ddbpp. Osservando gli ottimi proposti si nota che in poco tempo ci sono ``andati molto vicini'' seppure si notano comunque possibili margini di miglioramento. Pensando in un ottica di applicazione commerciale questo comportamento è favorevole in quanto risulta più facile ottimizzare un problema partendo già da una soluzione sub-ottima che in poco tempo è riuscita ad individuare pattern particolari.

Confrontando i due \emph{Core} a nostra disposizione la prima differenza che salta all'occhio è nel numero di paramentri: l'algoritmo \emph{genetico} richiede una configurazione più ampia che cambia da un'istanza all'altra, mentre il \emph{tabu search} offre parametri con validità più generale. Detto ciò il \emph{tabu search} risulta essere più intuitivo per la logica umana mentre il \emph{genetico} sembra più casuale nella sua evoluzione ma, dai test effettuati, risulta comportarsi meglio su istanze che coinvolgono pochi bin mentre i due algoritmi tendono a produrre soluzioni equivalenti con l'aumentare del numero di bin necessari.

In rete è possibile trovare programmi commerciali\footnote{In particolare abbiamo individuato il programma \emph{2D Load Packer} della Astrokettle Algorithms (\url{http://www.astrokettle.com/pr2dlp.html}) che forniva una libreria di istanze risolte, usate come metrica di riferimento.} che risolvono il problema del \ddbp, si è deciso quindi di confrontarsi con tali programmi su medesime istanze per valutare la bontà del codice prodotto, è emerso che per quasi tutte le istanze i risultati da noi ottenuti sono molti vicini a quelli ottenibili con i software commerciali. Questo evidenzia il fatto che con particolari accorgimenti potrebbe esser possibile migliorare leggermente quanto ottenuto, le tipologie di algoritmi per problemi quali il \ddbp sono infatti molto sensibili a particolari accorgimenti come ordinamenti o euristiche distinte adottate.

Il programma sviluppato è distribuito sotto licenza GNU GPLv3\footnote{\url{http://www.gnu.org/licenses/gpl.html}} e liberamente scaricabile dal sito \url{https://code.google.com/p/iaproject-2dbpp/}.